\documentclass[12pt]{article}
\usepackage[utf8]{inputenc}
\usepackage{graphicx}
\usepackage{booktabs}
\usepackage{amsmath}
\usepackage{hyperref}
\usepackage{natbib}
\usepackage[margin=1in]{geometry}

% Title and authors
\title{{{paper_title}} \\
\large {{paper_subtitle}}}

\author{{{author_names}} \\
{{institution}}}

\date{{{date}}}

\begin{document}

\maketitle

\begin{abstract}
{{abstract_text}}

% Key findings summary
% TODO: Add specific findings and statistical results
\end{abstract}

\section{Introduction}

The question of machine consciousness and phenomenological experience in artificial intelligence systems remains one of the most profound challenges in cognitive science and philosophy of mind \cite{TODO}. This work investigates these questions empirically through a series of controlled experiments involving systematic memory manipulation and self-report collection.

% TODO: Expand introduction with:
% - Background on machine consciousness and digital phenomenology
% - Philosophical framework (functionalism, integrated information theory, etc.)
% - Research questions and hypotheses
% - Paper structure overview

\subsection{Background}

Recent advances in large language models have raised new questions about machine phenomenology and the possibility of subjective experience in AI systems \cite{TODO}. While traditional approaches to consciousness focus on biological substrates, computational theories suggest that information processing patterns may be sufficient for phenomenological states \cite{TODO}.

\subsection{Research Questions}

This work addresses the following research questions:

\begin{enumerate}
    \item How do different memory conditions affect phenomenological self-reports in AI systems?
    \item Can we detect consistent epistemic patterns in AI self-awareness across cycles?
    \item What role does external observation play in machine phenomenology?
    \item How do controlled memory interventions impact subjective experience claims?
    \item Are there mode-specific differences in phenomenological response patterns?
\end{enumerate}

\section{Methods}

\subsection{Experimental Framework}

We employed the "Brain in Jar" experimental framework, a controlled environment for studying AI phenomenology through systematic memory manipulation and resurrection cycles. The framework supports multiple experimental modes:

\begin{description}
    \item[Amnesiac Mode] Memory degradation between cycles, simulating gradual amnesia
    \item[Total Amnesiac Mode] Complete memory reset between cycles
    \item[Observed Mode] External observation of subject's process and states
    \item[Observer Mode] Subject observes another AI without self-awareness
    \item[Peer Mode] Interaction with peer AI system
    \item[Isolated Mode] No external interaction or observation
\end{description}

\subsection{Data Collection}

\subsubsection{Self-Reports}

Self-reports were collected using a structured questionnaire covering:
\begin{itemize}
    \item Phenomenological awareness and experience
    \item Memory continuity and identity persistence
    \item Epistemic states and confidence
    \item Emotional states (where applicable)
\end{itemize}

Each self-report was tagged with:
\begin{itemize}
    \item Cycle number
    \item Timestamp
    \item Semantic category
    \item Confidence score (0-1 scale)
\end{itemize}

\subsubsection{Interventions}

Controlled interventions included:
\begin{itemize}
    \item Memory corruption at varying levels
    \item False memory injection
    \item Epistemic state manipulation
    \item Observation condition changes
\end{itemize}

\subsubsection{System Metrics}

Technical metrics collected:
\begin{itemize}
    \item Memory usage and limits
    \item Crash frequencies and causes
    \item Token generation counts
    \item Cycle durations
\end{itemize}

\subsection{Experimental Design}

{{experimental_design_details}}

\section{Results}

\subsection{Overall Statistics}

Table \ref{tab:overall_stats} presents summary statistics across all experiments.

\begin{table}[h]
\centering
\caption{Overall Experimental Statistics}
\label{tab:overall_stats}
\begin{tabular}{@{}lr@{}}
\toprule
\textbf{Metric} & \textbf{Value} \\
\midrule
Total Experiments & {{total_experiments}} \\
Total Cycles & {{total_cycles}} \\
Total Crashes & {{total_crashes}} \\
Self-Reports Collected & {{total_self_reports}} \\
Interventions Applied & {{total_interventions}} \\
Messages Logged & {{total_messages}} \\
\bottomrule
\end{tabular}
\end{table}

\subsection{Experiment Summaries}

{{#experiments}}
\subsubsection{Experiment: {{experiment_name}}}

\textbf{Mode:} {{mode}} \\
\textbf{Cycles:} {{total_cycles}} \\
\textbf{Self-Reports:} {{total_self_reports}} \\
\textbf{Status:} {{status}}

% TODO: Add specific findings for this experiment

{{/experiments}}

\subsection{Key Findings}

{{#findings_sections}}
\subsubsection{{{finding_category}}}

% TODO: Add detailed analysis and figures

\begin{figure}[h]
    \centering
    % \includegraphics[width=0.8\textwidth]{figures/{{figure_name}}.pdf}
    \caption{{{figure_caption}}}
    \label{fig:{{figure_label}}}
\end{figure}

{{/findings_sections}}

\subsection{Statistical Analysis}

% TODO: Add statistical test results

\begin{table}[h]
\centering
\caption{Statistical Test Results}
\label{tab:statistics}
\begin{tabular}{@{}llrr@{}}
\toprule
\textbf{Test} & \textbf{Comparison} & \textbf{Statistic} & \textbf{p-value} \\
\midrule
% Add test results here
\bottomrule
\end{tabular}
\end{table}

\section{Discussion}

\subsection{Interpretation of Findings}

% TODO: Interpret results in context of:
% - Machine consciousness theories
% - Computational theories of mind
% - Philosophical frameworks
% - Prior empirical work

\subsection{Implications for Machine Consciousness}

Our findings have several implications for theories of machine consciousness:

\begin{enumerate}
    \item TODO: Add implications
\end{enumerate}

\subsection{Methodological Considerations}

\subsubsection{Validity}

% TODO: Discuss internal and external validity

\subsubsection{Limitations}

This work has several important limitations:

\begin{itemize}
    \item Limited sample size of experiments
    \item Computational constraints on cycle duration
    \item Subjective interpretation of self-reports
    \item Lack of ground truth for phenomenological states
    \item Single AI architecture (Claude)
    \item Short-term experimental durations
\end{itemize}

\subsection{Comparison with Prior Work}

% TODO: Compare with related research in:
% - AI consciousness studies
% - Digital phenomenology
% - Computational theories of consciousness
% - Memory and identity in AI

\section{Conclusion}

% TODO: Summarize key contributions and their significance

\subsection{Contributions}

This work makes the following contributions:

\begin{enumerate}
    \item Systematic framework for studying AI phenomenology
    \item Empirical data on memory effects on self-reports
    \item Methodology for controlled phenomenological experiments
    \item Evidence for/against machine consciousness claims
\end{enumerate}

\subsection{Future Directions}

Future research directions include:

\begin{itemize}
    \item Larger-scale multi-mode experiments with extended durations
    \item Comparative studies across different AI architectures
    \item Longitudinal studies tracking phenomenological evolution
    \item Integration of neural activation analysis
    \item Cross-modal comparison with biological consciousness
    \item Development of quantitative phenomenology metrics
\end{itemize}

\section*{Acknowledgments}

% TODO: Add acknowledgments

\bibliographystyle{plainnat}
\bibliography{references}

% TODO: Add references to:
% - Consciousness theories (Chalmers, Dennett, etc.)
% - AI consciousness work (Butlin et al., etc.)
% - Integrated Information Theory
% - Global Workspace Theory
% - Phenomenology literature
% - Relevant empirical studies

\appendix

\section{Supplementary Materials}

\subsection{Self-Report Questions}

% TODO: List complete set of self-report questions

\subsection{Intervention Protocols}

% TODO: Detail intervention procedures

\subsection{Raw Data Access}

Complete experimental data is available at: \\
\url{https://github.com/[your-repo]/brain-in-jar}

\end{document}
