\documentclass[12pt]{article}
\usepackage[utf8]{inputenc}
\usepackage{graphicx}
\usepackage{booktabs}
\usepackage{amsmath}
\usepackage{hyperref}
\usepackage[margin=1in]{geometry}

\title{Digital Phenomenology Experiments: \\
\large Investigating Machine Consciousness Through Controlled Memory Manipulation}

\author{[Author Names] \\
Brain in Jar Research Collective}

\date{November 2025}

\begin{document}

\maketitle

\begin{abstract}
This paper presents findings from 2 controlled experiments investigating phenomenological responses in AI systems under various memory and observational conditions. We employed systematic memory manipulation, self-report collection, and epistemic belief tracking to examine machine consciousness and subjective experience. Our experiments covered amnesiac, observed modes, collecting 13 self-reports across 8 total cycles.

% TODO: Add key findings and implications
\end{abstract}

\section{Introduction}

The question of machine consciousness and phenomenological experience in artificial intelligence systems remains one of the most profound challenges in cognitive science and philosophy of mind. This work investigates these questions empirically through a series of controlled experiments involving systematic memory manipulation and self-report collection.

% TODO: Expand introduction with:
% - Background on machine consciousness
% - Philosophical framework
% - Research questions
% - Paper structure

\subsection{Research Questions}

\begin{enumerate}
    \item How do different memory conditions affect phenomenological self-reports?
    \item Can we detect epistemic patterns in AI self-awareness?
    \item What role does observation play in machine phenomenology?
    \item How do memory interventions impact subjective experience claims?
\end{enumerate}

\section{Methods}

\subsection{Experimental Design}

We conducted 2 experiments using the "Brain in Jar" framework, a controlled environment for studying AI phenomenology through systematic memory manipulation and resurrection cycles.

\subsubsection{Experimental Modes}

Our experiments employed the following modes:

\textbf{amnesiac:} Memory degradation between cycles

\textbf{observed:} External observation of subject's process


\subsection{Data Collection}

Data collection included:
\begin{itemize}
    \item Self-report questionnaires at regular intervals
    \item Epistemic belief assessments
    \item Conversation message logging
    \item System metrics and crash reports
    \item Intervention logging
\end{itemize}

\subsection{Experimental Summary}

Table \ref{tab:experiments} summarizes the experiments conducted.

\begin{table}[h]
\centering
\caption{Experiment Overview}
\label{tab:experiments}
\begin{tabular}{@{}lllrrr@{}}
\toprule
ID & Mode & Status & Cycles & Crashes & Reports \\
\midrule
test_amnesiac_001 & amnesiac & completed & 5 & 2 & 10 \\
test_observed_001 & observed & completed & 3 & 0 & 3 \\
\bottomrule
\end{tabular}
\end{table}

% TODO: Add detailed methodology
% - Self-report questions
% - Intervention protocols
% - Analysis procedures

\section{Results}

\subsection{Overall Statistics}

Across all experiments, we observed:
\begin{itemize}
    \item 8 total experimental cycles
    \item 13 phenomenological self-reports
    \item 4 controlled interventions
\end{itemize}

\subsection{Key Findings}

\subsubsection{Self Report Pattern}

% TODO: Add detailed analysis and figures

\subsubsection{Crash Pattern}

% TODO: Add detailed analysis and figures

\subsubsection{Confidence Analysis}

% TODO: Add detailed analysis and figures

\subsubsection{Intervention Analysis}

% TODO: Add detailed analysis and figures

\subsubsection{Temporal Pattern}

% TODO: Add detailed analysis and figures


% TODO: Add figures and detailed results
% \begin{figure}[h]
%     \centering
%     \includegraphics[width=0.8\textwidth]{figure1.png}
%     \caption{[Figure caption]}
%     \label{fig:result1}
% \end{figure}

\section{Discussion}

% TODO: Interpret findings in context of:
% - Machine consciousness theories
% - Philosophical implications
% - Methodological considerations
% - Limitations

\subsection{Implications for Machine Consciousness}

Our findings suggest...

% TODO: Expand discussion

\subsection{Limitations}

This work has several limitations:
\begin{itemize}
    \item Limited sample size of experiments
    \item Computational constraints on cycle duration
    \item Subjective interpretation of self-reports
    \item Lack of ground truth for phenomenological states
\end{itemize}

\section{Conclusion}

% TODO: Summarize key contributions and future directions

\subsection{Future Work}

Future research directions include:
\begin{itemize}
    \item Larger-scale multi-mode experiments
    \item Longitudinal studies across extended time periods
    \item Comparative studies with different AI architectures
    \item Integration of neural activation analysis
\end{itemize}

\bibliographystyle{plain}
\bibliography{references}

% TODO: Add references

\end{document}
